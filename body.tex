
\section{Education}
\cventry{2011--present}{PhD. High Energy Physics}{Imperial College London \& CERN}{}{}
{
  \begin{itemize}
    \item Thesis title:
      Searching for beyond the Standard Model physics using direct and indirect methods at
      LHCb
  \end{itemize}
}
\cventry{2007--2011}{First Class (MPhys) Theoretical Physics}
{University of Durham {\color{color2} St.~Aidan's College}}{}{}{}
\cventry{2004--2006}{Sixth Form College}{Hill's Road Sixth Form College {\color{color2} Cambridge}}{}{}
{Maths (A), Further Maths (A), Physics (A), Chemistry (A), History AS (A)}
\cventry{1999--2004}{Secondary school}{Comberton Village College {\color{color2} Cambridge}}{}{}
{GCSEs (6A$\!^*$ 2A 1B); GNVQ ICT (Dist.); AS-Level History (A)}

\section{Computing Skills}
\cvline{Languages}{\cpp, Python, bash scripting, MatLab}
\cvline{Software frameworks}{ROOT/RooFit/TMVA, numpy/scipy/matplotlib/pandas/sklearn}
\cvline{Tools}{Git, SVN, \LaTeX, MS Office}
\cvline{OS}{Linux, Windows}


\section{Employment History}
\cventry{2011--present}
{PhD research}
{Imperial College London and CERN, Geneva}
{}{}
{
  Researcher working on LHCb experiment at the Large Hadron Collider, two years (June 2012 -- June
  2014) were spent at CERN.
  Work involved the analysis of data collected by the detector using computing techniques including
  machine learning algorithms.
  Experience working both independently and in small groups, and frequently presented to the
  collaboration.
  Involved with the development of software package in the LHCb framework; designed for
  collaboration wide use.
  Monte Carlo simulations and statistical techniques applied to data to understand the physics
  involved.
}
\cventry{Oct 2014--Jan 2015}
{Student Teaching Assistant}
{Imperial College London}
{}{}
{
  \begin{itemize}
    \item Demonstrator in Level 1 Python course.
    \item Ensure students have understanding of the language and how to use it.
    \item Assess written reports using the techniques students learnt.
  \end{itemize}
}
\cventry{2006--2010}
{Technology Scholar}
{Cambridge Consultants}
{}{}
{
  Part of the Medical Electronics Group, which was previously Wireless ASICs (Application
  Specific ICs) often in projects with large groups of engineers.
  \begin{itemize}
    \item Full time employee in 2006--2007, worked summers in following years.
    \item Development of novel optical devices.
    \item Individual project working on a wrist-worn PPG device.
  \end{itemize}
}
\cventry{2004--2006}{Waitrose, Cambridge}{Caf\'e Staff}{}{}{}


\section{Other}
%\section{Interests}
\cvline{Driving}{Have a full, clean driving licence.}

\cventry{}{Other than software written for analysis of LHCb data, have authored wol.}{Software}{}{}{
  Wol is a command line manager of pdfs, particularly designed to organise arXiv papers.
  It is only a small project, but the tool is used by a handful of people to great effect.
}

%\cvline{Additional Qualifications}{Level 1 Squash Coach (2005).}
\cventry{}{Enjoy many sports, main interests are:}{Sports}{}{}{
  \begin{itemize}
    \item Squash.
      Captained my college team at Durham University.
      Played in the third team at Imperial.
      Became an England Squash Level 1 registered coach.
    \item Cycling.
      Part of the Imperial cycling club.
      Cycled from Geneva to London last summer ($\sim\!550$ miles, 6 days in the saddle).
    \item Skiing and Sailing.
      Have enjoyed these sports for many years, but did a great deal while living in Geneva.
  \end{itemize}
}

\cventry{}{Other hobbies include:}{Other activities}{}{}{
  \begin{itemize}
    \item Travel.
      Excited by exploring new places, spent the summer of my gap year travelling around South-East
      Asia and South America.
      Since then travelled to many places in groups and alone.
      Enjoy seeing new places.
    \item Hiking. Mercantour etc.
      Been on several hiking holidays, including a week trek in the Mercantour national park in
      France, in which we slept in hammocks.
    \item Photography. Keen amateur photographer.
  \end{itemize}
}



\section{Publications and Conference presentations}

\publication{Oct 2014}
{First observations of the rare decays
  $\boldsymbol{B^+\!\to K^+\pi^+\pi^-\mu^+\mu^-}$ and
  $\boldsymbol{B^+\!\to\phi K^+\mu^+\mu^-}$}
  {Journal of High Energy Physics \textbf{02}, 043 (2013), arXiv:1210.1089}{Publication}
  %{arXiv:1408.1137}{Publication}
\publication{Apr 2014}
{Towards measurements of CKM parameters with loops and trees at LHCb}
{Institute of Physics High Energy Particle and Astro Particle
Physics (London, UK)}{Conference}
\publication{Aug 2013}
{Rare decays at LHCb}
{International Conference on New Frontiers in Physics 2013 (Crete, Greece)}
{Conference}
\publication{Sept 2013}
{Search for rare decays of the form $\boldsymbol{B\to hhh\mu\mu}$}
{LHCb week (Krakow, Poland)}{Conference}
\publication{Oct 2012}
{First evidence for the annihilation decay mode $\boldsymbol{B^+\!\to D_s^+\phi}$}
{Journal of High Energy Physics \textbf{10}, 064 (2014), arXiv:1210.1089}{Publication}
%{arXiv:1210.1089}{Publication}







