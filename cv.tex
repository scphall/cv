\makecvtitle


\section{Profile}
\cvline{}{Physics, Mathematics, Programming and Data Analysis}
\cvline{}{Python and C++}
\cvline{}{}
{
  Doctoral student at Imperial College London with experience in the analysis of data
  from the LHCb detector, which is located on the Large Hadron Collider.
  Used a variety of computational and statistial techniques to analyse these datasets, and
  motivated by applying them to others.
  Good work ethic, problem solving skills, quick at adapting to new problems working as part of a
  team or independently.
  %quaick at adapting
  %team or independently
  %\begin{itemize}
    %\item Problem solving
    %\item Quick to pick up new software and programming languages
    %\item team / independently
  %\end{itemize}
}
%\cvline{Programming and Software Development}{Python and C++}
%{
  %\begin{itemize}
    %\item EvtGen
  %\end{itemize}
%}


\section{Education}
\cventry{2011--present}{PhD. High Energy Physics}{Imperial College London \& CERN}{}{}
{
  \begin{itemize}
    \item Thesis title:
    %\item Included two years based at the Large Hadron Collider at CERN, near Geneva.
    \item Details are given below.
  \end{itemize}
}
\cventry{2007--2011}{First Class (MPhys) Theoretical Physics}
{University of Durham {\color{color2} -- St.~Aidan's College}}{}{}{}
\cventry{2004--2006}{A-Levels}{Hill's Road Sixth Form College {\color{color2} -- Cambridge}}{}{}
{Maths (A), Further Maths (A), Physics (A), Chemistry (A), History AS (A)}
\cventry{1999--2004}{Secondary school}{Comberton Village College {\color{color2} -- Cambridge}}{}{}
{GCSEs (6A$^*$ 2A 1B); GNVQ ICT (Dist.); AS-Level History (A)}

\section{Computing Skills}
%\cvcomputer{Languages}{C++, Python, bash}{Software frameworks}{ROOT, RooFit, TMVA, numpy/scipy/matplotlib}
%\cvcomputer{OS}{Linux, Windows}{Tools}{Git, SVN, \LaTeX, MS Office}
\cvline{Languages}{C++, Python, bash scripting, MatLab}
\cvline{Software frameworks}{ROOT/RooFit/TMVA, numpy/scipy/matplotlib/pandas/sklearn}
\cvline{Tools}{Git, SVN, \LaTeX, MS Office}
\cvline{OS}{Linux, Windows}


\section{Empolyment History}
%\cventry{June 2012--June 2014}
\cventry{2011--present}
{PhD. research}
{Imperial College London and CERN, Geneva}
{}{}
{
  Researcher working on LHCb experiement at the Large Hadron Collider, two years (June 2012 -- June
  2014) were spent at CERN.
  Work involved the analysis of data collected by the detector using computing techniques including
  machine learning algorithms.
  Had experience working both independently and in small groups, and frequently presented to the
  collaboration.
  Involved with the development of software package in the LHCb framework designed for
  collaboration wide use.
  Used Monte Carlo simulations to understand physics data.
  Use of statistical techniques.
  %Main roles and responsibilities:
  %\begin{itemize}
    %\item Machine learning
    %\item Software framework
    %\item EvtGen
    %\item computing techniques to analyse data collected by lhcb at the lhc
    %\item data analysis
    %\item problen solving
    %\item presentations
    %\item MC simulations
    %\item Working in small group 2-3 people
    %\item statistical and systematic studies
  %\end{itemize}
}
\cventry{Oct 2014--Jan2015}
{Student Teaching Assistant}
{Imperial College London}
{}{}
{
  \begin{itemize}
    \item Teacing and demonstrating Level 1 python course.
    \item Ensuring students had an understanding of the language and how to use it.
    \item Assessed written reports which used the techniques students learnt.
  \end{itemize}
}
\cventry{2006--2010}
{Technology Scholar}
{Cambridge Consultants}
{}{}
{
  Worked in the Medical Electronics Group (previosly named Wireless ASICs --- Application Specific
  ICs) often in projects with large groups of engineers.
  \begin{itemize}
    \item Full time employee in 2006-2007, worked summers in following years.
    \item Development of novel optical devices
    \item Individual project
  \end{itemize}
}
\cventry{2004--2006}{Waitrose, Cambridge}{Caf\'e Staff}{}{}{}


\section{Other}
%\section{Interests}
\cvline{Driving}{Have a full, clean driving licence.}

\cventry{}{Other than software written for analysis of LHCb data, have authored wol.}{Software}{}{}{
  Wol is a command line manager of pdfs, particularly designed to organise arXiv papers.
  It is only a small project, but the tool is used by a handful of people to great effect.
}

%\cvline{Additional Qualifications}{Level 1 Squash Coach (2005).}
\cventry{}{Enjoy many sports, main interests are:}{Sports}{}{}{
  \begin{itemize}
    \item Squash.
      Captained my college team at Durham University.
      Played in the third team at Imperial.
      England Squash Level 1 registered coach.
    \item Cycling.
      Part of the Imperial cycling club.
      Cycled from Geneva to London last summer ($\sim550$ miles, 6 days in the saddle).
    \item Skiing and Sailing.
      Have enjoyed these sports for many years, but did a great deal whlie living in Geneva.
  \end{itemize}
}

\cventry{}{Other hobbies include:}{Other activities}{}{}{
  \begin{itemize}
    \item Travel.
      Excited by exploring new places, spent the summer of my gap year travelling around South-East
      Asia and South America.
      Since then travelled to many places in groups and alone.
      Enjoy seeing new places.
    \item Hiking.
    \item Photography.
  \end{itemize}
}



\section{Publucations and Conference presentations}

\publication{Aug 2014}
{First observations of the rare decays
  $\boldsymbol{B^+\!\to K^+\pi^+\pi^-\mu^+\mu^-}$ and
  $\boldsymbol{B^+\!\to\phi K^+\mu^+\mu^-}$}
  {arXiv:1408.1137}{Publication}
\publication{Apr 2014}
{Towards measurements of CKM parameters with loops and trees at LHCb}
{Institute of Physics High Energy Particle Physics and Astro Particle
Physics 2014 (London, UK)}{Conference}
\publication{Aug 2013}
{Rare decays at LHCb}
{International Conference on New Frontiers in Physics 2013 (Crete, Greece)}
{Conference}
\publication{Sept 2013}
{Search for rare decays of the form $\boldsymbol{B\to hhh\mu\mu}$}
{LHCb week (Krakow, Poland)}{Conference}
\publication{Oct 2012}
{First evidence for the annihilation deecay mode $\boldsymbol{B^+\!\to D_s^+\phi}$}
{arXiv:1210.1089}{Publication}






\section{References}
Available on request



